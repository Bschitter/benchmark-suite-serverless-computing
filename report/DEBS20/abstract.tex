%!TEX root = main.tex
%\nocite{Pellegrini_2019}
\begin{abstract}
A major trend for cloud providers is serverless computing, \vs{PM: can you explain in 1 sentence what serveless computing is?}.
Despite this increasing popularity, there is little data regarding the actual performances of the available platforms.
Specifically, it is difficult to test and benchmark such systems in a independent manner.
Practitioners and researchers are obliged to deploy their systems to take an informed decisions regarding the most convenient solution, both in terms of performance than costs.

This paper presents \sys, a modular architecture and proof-of-concept implementation of a benchmark suite for serveless computing platforms.
\sys currently supports the four major serverless cloud providers (\emph{i.e.}, AWS, Azure, Google, IBM) and a large set of different benchmark tests.
The suite fully automatizes the deployment, execution and clean-up of such tests, providing practitioners (historical) insights on the current performance observed by serverless applications. 
Additionally, we designed a pricing calculator to estimate to easily budgetize the deployment across the supported providers.
\sys is open-source and available at \sloppy{\url{https://github.com/bschitter/benchmark-suite-serverless-computing}}.
%Among all the services the cloud offers, serverless computing is very popular and every large cloud service provider offers a serverless platform. 
%However, there has not been much effort to test, benchmark or compare these services. 
%This thesis tries to take an approach on benchmarking serverless computing with building a suite that everyone can use.

%This suite provides five different tests to benchmark on four different clouds, highly automated deployment and cleanup of these tests, a testing utility for comparing cloud providers, a heavy benchmark to load test the serverless platforms and a pricing calculator to estimate hypothetical and actual costs. 
%Everything is packaged with Docker and is easy to use.

%This benchmark suite is completely open source and the code and the documentation can be found at \url{https://github.com/bschitter/benchmark-suite-serverless-computing}.
\end{abstract}
%\textbf{Keywords:} Serverless, Serverless Computing, Benchmarking, \gls{AWS} Lambda, Microsoft Azure Functions, Google Cloud Functions, %\gls{IBM} Cloud Functions, \gls{FaaS}
