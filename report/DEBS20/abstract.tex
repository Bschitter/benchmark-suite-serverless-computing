%!TEX root = main.tex
%\nocite{Pellegrini_2019}
\begin{abstract}
A major trend for cloud providers is serverless computing, where the responsibility of managing servers, allocation of resources, scaling and availability is fully delegated to the provider. %\mais{Ok?}.
Despite its increasing popularity, there is little data regarding the actual application performances achievable on the currently available platforms.
Specifically, it is cumbersome to benchmark such systems in an independent manner, and one must resort to a full deployment of a systems to take informed decisions on the most convenient solution, both in terms of performance and costs.

\sys is a modular architecture and proof-of-concept implementation of a benchmark suite for serveless computing platforms.
It currently supports the current mainstream serverless cloud providers (\emph{i.e.}, AWS, Azure, Google, IBM), a large set of benchmark tests and a variety of implementation languages.
The suite fully automatizes the deployment, execution and clean-up of such tests, providing practitioner insights (including historical ones) on the current performance potentially observed by serverless applications. 
Moreover, \sys embeds a pricing calculator to estimate budget costs for the deployment across the supported providers.
\sys is open-source and available at \sloppy{\url{https://github.com/bschitter/benchmark-suite-serverless-computing}}.
%Among all the services the cloud offers, serverless computing is very popular and every large cloud service provider offers a serverless platform. 
%However, there has not been much effort to test, benchmark or compare these services. 
%This thesis tries to take an approach on benchmarking serverless computing with building a suite that everyone can use.

%This suite provides five different tests to benchmark on four different clouds, highly automated deployment and cleanup of these tests, a testing utility for comparing cloud providers, a heavy benchmark to load test the serverless platforms and a pricing calculator to estimate hypothetical and actual costs. 
%Everything is packaged with Docker and is easy to use.

%This benchmark suite is completely open source and the code and the documentation can be found at \url{https://github.com/bschitter/benchmark-suite-serverless-computing}.
\end{abstract}
%\textbf{Keywords:} Serverless, Serverless Computing, Benchmarking, \gls{AWS} Lambda, Microsoft Azure Functions, Google Cloud Functions, %\gls{IBM} Cloud Functions, \gls{FaaS}
