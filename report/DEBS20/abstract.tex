%!TEX root = main.tex
%\nocite{Pellegrini_2019}
\begin{abstract}
A major trend for cloud providers is serverless computing. 
Developers fully delegate to the cloud provider the task of managing the servers, dynamically allocate the required resources, as well as handling availability and fault-tolerance matters.
In doing so, application developers solely focus on their application functions, which are then completely managed on the cloud.

Despite its increasing popularity, little data is known regarding the actual system performance achievable on the currently available serverless platforms.
Specifically, it is cumbersome to benchmark such systems in an languag- or runtime-independent manner.
Instead, one must resort to a full application deployment, to later take informed decisions on the most convenient solution along several dimensions, including performance and economic costs.

\sys is a modular architecture and proof-of-concept implementation of a benchmark suite for serveless computing platforms.
It currently supports the current mainstream serverless cloud providers (\emph{i.e.}, AWS, Azure, Google, IBM), a large set of benchmark tests and a variety of implementation languages.
The suite fully automatizes the deployment, execution and clean-up of such tests, providing insights (including historical ones) on the performance observed by serverless applications. 
We further provided \sys with a model to estimate budget costs for deployments across the supported providers.
\sys is open-source and available at \sloppy{\url{https://github.com/faasdom/benchmark-suite-serverless-computing}}.
%Among all the services the cloud offers, serverless computing is very popular and every large cloud service provider offers a serverless platform. 
%However, there has not been much effort to test, benchmark or compare these services. 
%This thesis tries to take an approach on benchmarking serverless computing with building a suite that everyone can use.

%This suite provides five different tests to benchmark on four different clouds, highly automated deployment and cleanup of these tests, a testing utility for comparing cloud providers, a heavy benchmark to load test the serverless platforms and a pricing calculator to estimate hypothetical and actual costs. 
%Everything is packaged with Docker and is easy to use.

%This benchmark suite is completely open source and the code and the documentation can be found at \url{https://github.com/bschitter/benchmark-suite-serverless-computing}.
\end{abstract}
%\textbf{Keywords:} Serverless, Serverless Computing, Benchmarking, \gls{AWS} Lambda, Microsoft Azure Functions, Google Cloud Functions, %\gls{IBM} Cloud Functions, \gls{FaaS}
