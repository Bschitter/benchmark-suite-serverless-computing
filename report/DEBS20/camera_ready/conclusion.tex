%!TEX root = main.tex

\section{Conclusion and Future Work}
\label{sec:conclusion}

Motivated by the lack of common FaaS framework or standard platform to evaluate serverless platforms, this paper presented \sys, a user-space application that facilitates performance testing across a variety of serverless platform.
\sys is an open-source benchmark suite that is easy to deploy and provides meaningful insights across a variety of metrics and serverless providers. 

We envision to extend \sys along the following directions.
First, we will expand the set of supported languages and runtime systems, also by integrating contributions from the open-source community.
Second, we plan to maintain a continuous, online deployment of the \sys, publicly accessible, in order to build a larger dataset of historical measurements that will be released to the research community.
We finally intend to include native support for Docker images to ship functions, in order to facilitate integration with services such as Google Cloud Run~\cite{cloudrun}.

%Although this benchmark suite covers the most important aspects it can be developed and be improved further. Apart from general improvements and optimizations more runtimes respectively languages and cloud providers could be supported, more different tests could be implemented and provided. 
%In addition, a plotting integration (e.g. with R) would be useful as Grafana does not provide much exporting and plotting capabilities and is rather focused on live monitoring.\\
%There could be a test regarding continuous deployment and inspect the behaviour of the platform. Besides, a special focus could lay on deploying and testing Docker images because that might be an important aspect and trend in the future, considering there are no runtime restrictions of the provider. Several providers already offer to deploy just a Docker image (AWS, Azure, IBM) and Google has its service Cloud Run which has become generally available on November 14, 2019 \cite{cloudrun} and is a fully managed, serverless container platform.\\ 
%Moreover, the load test should probably be carried out with different function execution times to get a better understanding of the possible connection between scaling and execution time, if there is any at all. 
%Also, a real world application test would be beneficial and demonstrate the usability of serverless computing.

\newpage